%%
%% This is file `sample-sigchi.tex',
%% generated with the docstrip utility.
%%
%% The original source files were:
%%
%% samples.dtx  (with options: `sigchi')
%% 
%% IMPORTANT NOTICE:
%% 
%% For the copyright see the source file.
%% 
%% Any modified versions of this file must be renamed
%% with new filenames distinct from sample-sigchi.tex.
%% 
%% For distribution of the original source see the terms
%% for copying and modification in the file samples.dtx.
%% 
%% This generated file may be distributed as long as the
%% original source files, as listed above, are part of the
%% same distribution. (The sources need not necessarily be
%% in the same archive or directory.)
%%
%% The first command in your LaTeX source must be the \documentclass command.
\documentclass[sigplan]{acmart}
\usepackage{booktabs}
%%
%% \BibTeX command to typeset BibTeX logo in the docs
\AtBeginDocument{%
  \providecommand\BibTeX{{%
    \normalfont B\kern-0.5em{\scshape i\kern-0.25em b}\kern-0.8em\TeX}}}

%% Rights management information.  This information is sent to you
%% when you complete the rights form.  These commands have SAMPLE
%% values in them; it is your responsibility as an author to replace
%% the commands and values with those provided to you when you
%% complete the rights form.
\setcopyright{acmcopyright}
\copyrightyear{2018}
\acmYear{2018}
\acmDOI{10.1145/1122445.1122456}

%% These commands are for a PROCEEDINGS abstract or paper.
\acmConference[Woodstock '18]{Woodstock '18: ACM Symposium on Neural
  Gaze Detection}{June 03--05, 2018}{Woodstock, NY}
\acmBooktitle{Woodstock '18: ACM Symposium on Neural Gaze Detection,
  June 03--05, 2018, Woodstock, NY}
\acmPrice{15.00}
\acmISBN{978-1-4503-XXXX-X/18/06}


%%
%% Submission ID.
%% Use this when submitting an article to a sponsored event. You'll
%% receive a unique submission ID from the organizers
%% of the event, and this ID should be used as the parameter to this command.
%%\acmSubmissionID{123-A56-BU3}

%%
%% The majority of ACM publications use numbered citations and
%% references.  The command \citestyle{authoryear} switches to the
%% "author year" style.
%%
%% If you are preparing content for an event
%% sponsored by ACM SIGGRAPH, you must use the "author year" style of
%% citations and references.
%% Uncommenting
%% the next command will enable that style.
%%\citestyle{acmauthoryear}

\newcommand{\theArticle}{\textit{Towards innovation measurement in the software industry}}
%%
%% end of the preamble, start of the body of the document source.
\begin{document}

%%
%% The "title" command has an optional parameter,
%% allowing the author to define a "short title" to be used in page headers.
\title{The impact of a proposal for innovation measurement in the software industry}

%%
%% The "author" command and its associated commands are used to define
%% the authors and their affiliations.
%% Of note is the shared affiliation of the first two authors, and the
%% "authornote" and "authornotemark" commands
%% used to denote shared contribution to the research.
\author{Nauman bin Ali}
\orcid{0000-0001-7266-5632}
\affiliation{%
  \institution{Blekinge Institute of Technology}
  \country{Sweden}}
\email{nal@bth.se}

\author{Henry Edison}
\affiliation{%
  \institution{Lero, NUI Galway}
  \country{Ireland}}
\email{henry.edison@nuigalway.ie}

\author{Richard Torkar}
 \orcid{0000-0002-0118-8143}
 \affiliation{
 \institution{Chalmers and University of Gothenburg}
 \country{Sweden}
 }
 \affiliation{
 \institution{Stellenbosch Institute for Advanced Study (STIAS)}
 \country{South Africa}
 }
 \email{torkarr@chalmers.se}


%%
%% By default, the full list of authors will be used in the page
%% headers. Often, this list is too long, and will overlap
%% other information printed in the page headers. This command allows
%% the author to define a more concise list
%% of authors' names for this purpose.
\renewcommand{\shortauthors}{Ali et al.}

%%
%% The abstract is a short summary of the work to be presented in the
%% article.



\begin{abstract}
	Measuring an organization's capability to innovate and assessing its innovation output and performance on the market is a challenging task. We proposed a comprehensive model and a suite of measurements to support this task. In the current paper, we have reflected on the impact of the work. We have mainly relied on quantitative and qualitative analysis of the citations of the paper.  
\end{abstract}

%%
%% The code below is generated by the tool at http://dl.acm.org/ccs.cfm.
%% Please copy and paste the code instead of the example below.
%%
\begin{CCSXML}
<ccs2012>
 <concept>
  <concept_id>10010520.10010553.10010562</concept_id>
  <concept_desc>Computer systems organization~Embedded systems</concept_desc>
  <concept_significance>500</concept_significance>
 </concept>
 <concept>
  <concept_id>10010520.10010575.10010755</concept_id>
  <concept_desc>Computer systems organization~Redundancy</concept_desc>
  <concept_significance>300</concept_significance>
 </concept>
 <concept>
  <concept_id>10010520.10010553.10010554</concept_id>
  <concept_desc>Computer systems organization~Robotics</concept_desc>
  <concept_significance>100</concept_significance>
 </concept>
 <concept>
  <concept_id>10003033.10003083.10003095</concept_id>
  <concept_desc>Networks~Network reliability</concept_desc>
  <concept_significance>100</concept_significance>
 </concept>
</ccs2012>
\end{CCSXML}

\ccsdesc[500]{Computer systems organization~Embedded systems}
\ccsdesc[300]{Computer systems organization~Redundancy}
\ccsdesc{Computer systems organization~Robotics}
\ccsdesc[100]{Networks~Network reliability}

%%
%% Keywords. The author(s) should pick words that accurately describe
%% the work being presented. Separate the keywords with commas.
\keywords{innovation, impact, relevance, measurement}


%%
%% This command processes the author and affiliation and title
%% information and builds the first part of the formatted document.
\maketitle

\section{Introduction}\label{sec:intro}
Innovation measurement in SE was a challenge---we contributed with a measurement framework in \theArticle~\cite{EdisonAT13}. 

The paper is structured as follows: Section \ref{sec:sumpaper} summarizes the contribution of \theArticle. In Section \ref{sec:whocites}, we describe a content analysis of the articles citing \theArticle. Section \ref{sec:soa} discusses the research identified in Section \ref{sec:whocites} that has extended our work. In Section \ref{sec:impact}, we discuss the research which documents the use of our work in industrial settings. Lastly, in Section \ref{sec:fw} we suggest some directions for future research.

\section{Summary and main contributions of \theArticle}\label{sec:sumpaper}
What was done.

\section{Overview of the papers citing our article}\label{sec:whocites}
Who found it relevant (would be good to have some qualitative data on how they ref the paper). Why did so many cite it?

For understanding the impact of our work, we have relied on the classification schema for academic citations proposed by \citet{teufel2006annotation}. The categories in their schema are listed and briefly described in Table~\ref{tab:CitationCategories}.

\begin{table*}[]
	\caption{Categories of citing papers from \citet{teufel2006annotation}}\label{tab:CitationCategories}
	\begin{tabular}{llp{12cm}}
		\toprule
		\textbf{Category}            & \textbf{Sub-category }& \textbf{Description}                                                                                                            \\ \midrule
		Weakness            & Weak         & Weakness of the approach pursued in \theArticle, Weakness in the definition, model, entities, attributes, or measurements of innovation as proposed in \theArticle                                                                                              \\
		\midrule
		Contrast/Comparison & CoCoGM       & Contrast/Comparison in Goals or Methods (neutral)                                                                    \\
		& CoCoR0       & Contrast/Comparison in Results (neutral)                                                                               \\
		& CoCo-        & Unfavourable Contrast/Comparison (current work is better than cited work)                                            \\
		& CoCoXY       & Contrast between two cited methods                                                                                     \\
		
		\midrule
		Positive sentiment  & PBas         & author uses cited work as starting point                                                                               \\
		& PUse         & author uses definitions/models/measures                                                                                      \\
		& PIUse\footnote{We have added this category}         & author uses cited work in industrial settings \\
		& PModi        & author adapts or modifies definition/model/measurements                                                                                 \\
		& PMot         & this citation is positive about approach or problem addressed (used to motivate work in current paper)                 \\
		& PSim         & author’s work and cited work are similar                                                                               \\
		& PSup         & author’s work and cited work are compatible/provide support for each other                                           \\
		
		\midrule
		Neutral             & Neut         & Neutral description of cited work, or not enough textual evidence for above categories or unlisted citation function.\\
		\bottomrule
	\end{tabular}
\end{table*}

On February 24, 2020, the \theArticle  had over 72 citations in Science Direct and Scopus, 61 in Web of Science Core Collection, and 234 in Google Scholar. To get a relatively complete picture of how this work has impacted further research, we decided to analyse the 234 citations on Google Scholar. 

\begin{table*}
	\caption{Results of an analysis of the citing papers}\label{tab:CitationAnalysis}
	\begin{tabular}{llllllll}
		\toprule
%		\textbf{Category}            &
		\textbf{Sub-category }& \textbf{References} & \textbf{Self-citations} & \textbf{Our network} & \textbf{From SE} & \textbf{Outside SE} & \textbf{Peer-reviewed} & \textbf{By practitioners} \\                                                                                                          \midrule
		 Weak         &                                                                                           \\
		\midrule
		CoCoGM       &                                                           \\
		CoCoR0 \\
		CoCo-        & \\
		CoCoXY       & \\
		\midrule
		
		PBas         & \\
		PUse         & \\
		PIUse         & \\
		PModi        &\\
		PMot         & \\
		PSim         & \\
		PSup         &  \\
		
		\midrule
		Neut         &\\
		\bottomrule
	\end{tabular}
\end{table*}

While discussing the citations the following reference will be useful \cite{penders2018ten} We can use this to also articulate why we have relied on citations as a way to reflect on the paper.


\section{Positioning in consideration of the recent state of the art and practice}\label{sec:soa}
What has been done after this (partly we'll get it from the previous section).

\section{Expected impact}\label{sec:impact}
Here it would be nice to show cases in industry not counting Ericsson. Perhaps we can get it from Section~\ref{sec:whocites}?

\section{New emerging ideas and current vision}\label{sec:fw}
What will be done, possibly


\section{References}

%%
%% The acknowledgments section is defined using the "acks" environment
%% (and NOT an unnumbered section). This ensures the proper
%% identification of the section in the article metadata, and the
%% consistent spelling of the heading.
%\begin{acks}
%\end{acks}



%%
%% The next two lines define the bibliography style to be used, and
%% the bibliography file.
\bibliographystyle{ACM-Reference-Format}
\bibliography{sample-base}

%%
%% If your work has an appendix, this is the place to put it.
%\appendix

%\section{Research Methods}


\end{document}
\endinput
%%
%% End of file `sample-sigchi.tex'.
%There are some more interesting ideas for inspiration in the CFP for SANER https://saner2020.csd.uwo.ca/eratrack