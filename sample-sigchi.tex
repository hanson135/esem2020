%%
%% This is file `sample-sigchi.tex',
%% generated with the docstrip utility.
%%
%% The original source files were:
%%
%% samples.dtx  (with options: `sigchi')
%% 
%% IMPORTANT NOTICE:
%% 
%% For the copyright see the source file.
%% 
%% Any modified versions of this file must be renamed
%% with new filenames distinct from sample-sigchi.tex.
%% 
%% For distribution of the original source see the terms
%% for copying and modification in the file samples.dtx.
%% 
%% This generated file may be distributed as long as the
%% original source files, as listed above, are part of the
%% same distribution. (The sources need not necessarily be
%% in the same archive or directory.)
%%
%% The first command in your LaTeX source must be the \documentclass command.
\documentclass[sigplan]{acmart}
\usepackage{booktabs}
\usepackage{todonotes}
%%
%% \BibTeX command to typeset BibTeX logo in the docs
\AtBeginDocument{%
  \providecommand\BibTeX{{%
    \normalfont B\kern-0.5em{\scshape i\kern-0.25em b}\kern-0.8em\TeX}}}

%% Rights management information.  This information is sent to you
%% when you complete the rights form.  These commands have SAMPLE
%% values in them; it is your responsibility as an author to replace
%% the commands and values with those provided to you when you
%% complete the rights form.
\setcopyright{acmcopyright}
\copyrightyear{2018}
\acmYear{2018}
\acmDOI{10.1145/1122445.1122456}

%% These commands are for a PROCEEDINGS abstract or paper.
\acmConference[Woodstock '18]{Woodstock '18: ACM Symposium on Neural
  Gaze Detection}{June 03--05, 2018}{Woodstock, NY}
\acmBooktitle{Woodstock '18: ACM Symposium on Neural Gaze Detection,
  June 03--05, 2018, Woodstock, NY}
\acmPrice{15.00}
\acmISBN{978-1-4503-XXXX-X/18/06}


%%
%% Submission ID.
%% Use this when submitting an article to a sponsored event. You'll
%% receive a unique submission ID from the organizers
%% of the event, and this ID should be used as the parameter to this command.
%%\acmSubmissionID{123-A56-BU3}

%%
%% The majority of ACM publications use numbered citations and
%% references.  The command \citestyle{authoryear} switches to the
%% "author year" style.
%%
%% If you are preparing content for an event
%% sponsored by ACM SIGGRAPH, you must use the "author year" style of
%% citations and references.
%% Uncommenting
%% the next command will enable that style.
%%\citestyle{acmauthoryear}

\newcommand{\theArticle}{\textit{Towards innovation measurement in the software industry}}
%%
%% end of the preamble, start of the body of the document source.
\begin{document}

%%
%% The "title" command has an optional parameter,
%% allowing the author to define a "short title" to be used in page headers.
\title{The impact of a proposal for innovation measurement in the software industry}

%%
%% The "author" command and its associated commands are used to define
%% the authors and their affiliations.
%% Of note is the shared affiliation of the first two authors, and the
%% "authornote" and "authornotemark" commands
%% used to denote shared contribution to the research.
\author{Nauman bin Ali}
\orcid{0000-0001-7266-5632}
\affiliation{%
  \institution{Blekinge Institute of Technology}
  \country{Sweden}}
\email{nal@bth.se}

\author{Henry Edison}
\affiliation{%
  \institution{Lero, NUI Galway}
  \country{Ireland}}
\email{henry.edison@nuigalway.ie}

\author{Richard Torkar}
 \orcid{0000-0002-0118-8143}
 \affiliation{
 \institution{Chalmers and University of Gothenburg}
 \country{Sweden}
 }
 \affiliation{
 \institution{Stellenbosch Institute for Advanced Study (STIAS)}
 \country{South Africa}
 }
 \email{torkarr@chalmers.se}


%%
%% By default, the full list of authors will be used in the page
%% headers. Often, this list is too long, and will overlap
%% other information printed in the page headers. This command allows
%% the author to define a more concise list
%% of authors' names for this purpose.
\renewcommand{\shortauthors}{Ali et al.}

%%
%% The abstract is a short summary of the work to be presented in the
%% article.



\begin{abstract}
Measuring an organization's capability to innovate and assessing its innovation output and performance on the market is a challenging task. We proposed a comprehensive model and a suite of measurements to support this task. In the current paper, we have reflected on the impact of the work. We have mainly relied on quantitative and qualitative analysis of the citations of the paper.  
\end{abstract}

%%
%% The code below is generated by the tool at http://dl.acm.org/ccs.cfm.
%% Please copy and paste the code instead of the example below.
%%
\begin{CCSXML}
<ccs2012>
 <concept>
  <concept_id>10010520.10010553.10010562</concept_id>
  <concept_desc>Computer systems organization~Embedded systems</concept_desc>
  <concept_significance>500</concept_significance>
 </concept>
 <concept>
  <concept_id>10010520.10010575.10010755</concept_id>
  <concept_desc>Computer systems organization~Redundancy</concept_desc>
  <concept_significance>300</concept_significance>
 </concept>
 <concept>
  <concept_id>10010520.10010553.10010554</concept_id>
  <concept_desc>Computer systems organization~Robotics</concept_desc>
  <concept_significance>100</concept_significance>
 </concept>
 <concept>
  <concept_id>10003033.10003083.10003095</concept_id>
  <concept_desc>Networks~Network reliability</concept_desc>
  <concept_significance>100</concept_significance>
 </concept>
</ccs2012>
\end{CCSXML}

\ccsdesc[500]{Computer systems organization~Embedded systems}
\ccsdesc[300]{Computer systems organization~Redundancy}
\ccsdesc{Computer systems organization~Robotics}
\ccsdesc[100]{Networks~Network reliability}

%%
%% Keywords. The author(s) should pick words that accurately describe
%% the work being presented. Separate the keywords with commas.
\keywords{innovation, impact, relevance, measurement}


%%
%% This command processes the author and affiliation and title
%% information and builds the first part of the formatted document.
\maketitle


\section{Introduction}\label{sec:intro}\todo {Richard}
Innovation measurement in SE was a challenge---we contributed with a measurement framework in \theArticle~\cite{EdisonAT13}. 

The paper is structured as follows: Section \ref{sec:sumpaper} summarizes the contribution of \theArticle. In Section \ref{sec:whocites}, we describe a content analysis of the articles citing \theArticle. Section \ref{sec:soa} discusses the research identified in Section \ref{sec:whocites} that has extended our work. In Section \ref{sec:impact}, we discuss the research which documents the use of our work in industrial settings. Section \ref{sec:fw} concludes the paper with some suggested directions for future research.

\section{Summary and main contributions of \theArticle}\label{sec:sumpaper}\todo{Henry}
Over the past, companies relied on cost and lead time reduction and quality improvement to strengthening their competitiveness. While quality is necessity, today it is not sufficient. Companies must continuously innovate; develop new processes and deliver new products to achieve and sustain a competitive advantage. Otherwise, they tend to lose their position to new and emerging startups that have innovative offerings. Such turnover signifies the importance of sustained innovation, thus the problem is not happen-stance innovation but rather doing it continuously on a regular basis. For sustained innovation to become a reality, a better understanding of innovation is required, which is possible only when innovation is measured.

The important of innovation measurement is well emphasised in industry. The Boston Consulting Group's survey \cite{andrew08} revealed that most executives believe that their companies should measure innovation as rigorously as core business operation, but less than half of companies actually do so. There is little consensus on how innovation measurement should be carried out. Each definition of innovation that is used signify a different aspect of innovation, e.g. perspectives, levels and types etc. This in turn determines what is considered as elements of innovation and how these are measured.

Organisations require means not only to measure their innovative output but also to assess their ability and capacity to innovate. Measurement helps to better understand and evaluate the consequences of the initiatives geared towards innovation. Furthermore,  like any other measurements, these will allow organisations to specify realistic targets of innovation in the future and to identify and resolve problems hindering progress towards goals, making. decisions and continuously improve the abilities to innovate.

The aims of this study were to establish the state of the art of innovation measurement and to capture the state of the practice of innovation measurement in the software industry. A systematic literature review (SLR) \cite{kitchenham07}  was conducted to establish the state of the art of innovation measurement, followed by a web-based questionnaire \cite{kasunic05} and face-to-face interviews \cite{creswell09} to collect the opinions of software industry practitioners and academics. In total, we retrieved 13,401 articles from seven digital libraries (Compendex, Scopus, IEEEXplore, ACM Digital Library, ScienceDirect and Business Source Premiere). After applying inclusion/exclusion criteria, 204 papers were accepted as the primary studies. We had 145 respondents out of which 94 completed the questionnaire - thus the completion rate was 54.83\%. Four industry practitioners (middle managers) and three academics with close relationship with industry were interviewed in this study. 

Our review shows that there are 41 definitions of innovation found in the literature which highlight 4 important attributes to measure: 
\begin{itemize}
	\item Impact of innovation on the market and technology, e.g. incremental or radical innovation, market or technological breakthrough.
	\item Types of innovation, e.g. product (new or significantly improved products), process (new or significantly improved design, analysis, or development method), market (new or significantly improved marketing methods, strategies, and concept in product design or packaging, placement, promotion, or pricing), and organisation innovation (new or significantly improved organisation methods, e.g. business practices, workplace organisation or external relations.
	\item Degree of novelty, e.g. new to the firm, new to the market, new to the world, and new to the industry.
	\item Nature of process: iterative process.
\end{itemize}

While twenty-eight determinants of innovation have been reported in literature, but only seven of them are studied in the software industry: internal collaboration, customer orientation, champions, human resources, strategy, networking, and leadership. Two-hundred and thirty-two metrics have been used to measure innovation in firm (88\%), industry (1\%), regional level (11\%). However, only 37\% of them have been statistically validated and 58\% have never been used in practice. Our review also identifies 13 innovation measurement frameworks. Most of these framework focus on technological breakthrough (8 frameworks). Out of these frameworks, only one framework have been studied in software companies. asds

\section{Methodology}\label{sec:method} \todo{Nauman}
For understanding the impact of \theArticle, we have relied on the classification schema for academic citations proposed by \citet{teufel2006annotation}. We also considered the taxonomy proposed by \citet{bornmann2008citation}. However, based on a pilot application we found \citet{teufel2006annotation} more straight forward and sufficient for our analysis. The decision is further supported by prior experience of using Bornmann and Daniel's taxonomy in software engineering literature \cite{poulding2015using}.

The categories in the schema we used are listed and briefly described in Table~\ref{tab:CitationCategories}. To separate any industrial application of our work we added a separate category. 

On February 24, 2020, the \theArticle  had over 72 citations in Science Direct and Scopus, 61 in Web of Science Core Collection, and 234 in Google Scholar. To get a relatively complete picture of how this work has impacted further research, we decided to analyse the 234 citations on Google Scholar. 

In a pilot, the first two authors classified ten randomly selected articles and discussed the use of categories. Thereafter, they divided the 234 articles among them and  independently classified them. The procedure followed is briefly summarized below: 
\begin{itemize}
\item Exclude citations where the full-text is not available. 
\item Exclude articles which are not written in English.
\item Exclude articles  that do not cite \theArticle in the full-text.
\item From the title, abstract and the publication venue judge the discipline of the publication (e.g. software industry, manufacturing, farming or automotive).
\item Only for conference papers and journal article, search for the citation to \theArticle in the full text, for each citation in the paper  read the entire paragraph containing it to understand the context, then classify the citation based on categories in Table \ref{tab:CitationCategories}.
\end{itemize}
 


\begin{table*}
	\caption{Categories of citing papers from \citet{teufel2006annotation}}\label{tab:CitationCategories}
	\begin{tabular}{llp{12cm}}
		\toprule
		\textbf{Category}            & \textbf{Sub-category }& \textbf{Description}                                                                                                            \\ \midrule
		Weakness            & Weak         & Weakness of the approach pursued in \theArticle, Weakness in the definition, model, entities, attributes, or measurements of innovation as proposed in \theArticle                                                                                              \\
		\midrule
		Contrast/Comparison & CoCoGM       & Contrast/Comparison in Goals or Methods (neutral)                                                                    \\
		& CoCoR0       & Contrast/Comparison in Results (neutral)                                                                               \\
		& CoCo-        & Unfavourable Contrast/Comparison (current work is better than the work in \theArticle)                                            \\
		& CoCoXY       & Contrast between a cited method and the method in \theArticle                                                                                     \\
		
		\midrule
		Positive sentiment  & PBas         & author uses the work in \theArticle as a starting point                                                                               \\
		& PUse         & author uses definitions/models/measures                                                                                      \\
		& PIUse\footnote{We have added this category}         & author uses the work in \theArticle in industrial settings \\
		& PModi        & author adapts or modifies definition/model/measurements  presented in \theArticle                                                                            \\
		& PMot         & this citation is positive about approach or problem addressed in \theArticle (used to motivate work in current paper)                 \\
		& PSim         & author’s work and the work in \theArticle are similar                                                                               \\
		& PSup         & author’s work and the work in \theArticle are compatible/provide support for each other                                           \\
		
		\midrule
		Neutral             & Neut         & Neutral description of cited work, or not enough textual evidence for above categories.\\
		\bottomrule
	\end{tabular}
\end{table*}

\section{Overview of the papers citing \theArticle}\label{sec:whocites} \todo{Nauman} %covers: Who found it relevant (would be good to have some qualitative data on how they ref the paper). Why did so many cite it?}

The 234 citations to\theArticle  were analysed using the process described in the Section \ref{sec:method}. 64 
Exclude papers 64 (52  were not written in English, 6 were inaccessible in full-text, 5 did not cite \theArticle in the body of the paper and 1 was a duplicate citation).

Grey literature:  53 citations are from what we have classified as grey  literature. Of these 53, 2 are  technical reports, 10 are book chapters and 41 are theses.

In total there are 108 conference papers  and 76 journal articles citing \theArticle. The analysis of their use of \theArticle is summarized in Table \ref{tab:CitationAnalysis}. 

The paper has nine self citations (including one from a thesis). 

When looking at the literature, where there is no stated connection to the context of software industry we see that the literature encompases several diverse fields including the following: automotive, banking, economics, farming, forestry, health sector, human resources, logistics, manufacturing, mechatronics, NGOs, oil industry, politics , restaurants and transportation. 



\begin{table*}
	\caption{Results of an analysis of the citing papers}
	\label{tab:CitationAnalysis}
	\begin{tabular}{p{2.5cm}llp{1.5cm}llll}
			\toprule
		& Total & Weak & Comparison / Contrast & Positive                                            & Neutral & Jrnl. & Conf. \\
		\midrule
		&       &      &                       &                                                     &         &         &            \\
		Self citations               & 9     & 0    & 0                     & 2 (PBas:1, PMot:1, PModi:1)                         & 6       & 5       & 1          \\
		From software related fields & 44    & 0    & 0                     & 17 (PBas:4, PModi:2,PUse:7,PMoti:4, PSup:1)         & 27      & 24      & 20         \\
		Others                       & 72    & 0    & 2                     & 21 (PBas:2, PModi:2,PUse:14,PMoti:2,PSim:1, PSup:2) & 48      & 57      & 15         \\
		Total                        & 116   & 0    & 2                     & 38 (PBas:6, PModi:2,PUse:21,PMoti:6,PSim:1, PSup:3) & 75      & 81      & 35        \\ \bottomrule
	\end{tabular}
\end{table*}

%
%\begin{table*}
%	\caption{Results of an analysis of the citing papers}\label{tab:CitationAnalysis}
%	\begin{tabular}{llllllll}
%		\toprule
%%		\textbf{Category}            &
%		\textbf{Sub-category }& \textbf{Refs.} & \textbf{Self-citations} & \textbf{Our network} & \textbf{From SE} & \textbf{Outside SE} & \textbf{Peer-reviewed} & \textbf{By practitioners} \\                                                                                                          \midrule
%		 Weak         &                                                                                           \\
%		\midrule
%		CoCoGM       &                                                           \\
%		CoCoR0 \\
%		CoCo-        & \\
%		CoCoXY       & \\
%		\midrule
%		
%		PBas         & \\
%		PUse         & \\
%		PIUse         & \\
%		PModi        &\\
%		PMot         & \\
%		PSim         & \\
%		PSup         &  \\
%		
%		\midrule
%		Neut         &\\
%		\bottomrule
%	\end{tabular}
%\end{table*}

While discussing the citations the following reference will be useful \cite{penders2018ten} We can use this to also articulate why we have relied on citations as a way to reflect on the paper.


\section{Positioning in consideration of the recent state of the art and practice}\label{sec:soa} \todo{Henry}
What has been done after this (partly we'll get it from the previous section).
Open innovation seems to be the area in SE that has been a follow-up of our work.

\section{Expected impact}\label{sec:impact}
Here it would be nice to show cases in industry not counting Ericsson. Perhaps we can get it from Section~\ref{sec:whocites}?



\section{New emerging ideas and current vision}\label{sec:fw} \todo{Possibly Richard?}
What will be done, possibly

\section*{Acknowledgement}
This project has received funding from the European Union's Horizon 2020 research and innovation programme. under the Marie Skłodowska-Curie grant agreement No. 754489 and with the financial support of the Science Foundation Ireland grant 13/RC/2094.

\section*{References}

%%
%% The acknowledgments section is defined using the "acks" environment
%% (and NOT an unnumbered section). This ensures the proper
%% identification of the section in the article metadata, and the
%% consistent spelling of the heading.
%\begin{acks}
%\end{acks}



%%
%% The next two lines define the bibliography style to be used, and
%% the bibliography file.
\bibliographystyle{ACM-Reference-Format}
\bibliography{sample-base}

%%
%% If your work has an appendix, this is the place to put it.
%\appendix

%\section{Research Methods}


\end{document}
\endinput
%%
%% End of file `sample-sigchi.tex'.
%There are some more interesting ideas for inspiration in the CFP for SANER https://saner2020.csd.uwo.ca/eratrack